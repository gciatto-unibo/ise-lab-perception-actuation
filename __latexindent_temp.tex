%%%%%%%%%%%%%%%%%%%%%%%%%%%%%%%%%%%%%%%%%%%%%%%%%%%%%%%%%%%%%%%%%%%%%%%%%%%%%%%%
% ISE Lab -- Topic
% Giovanni Ciatto
% Alma Mater Studiorum - Università di Bologna
% mailto:giovanni.ciatto@unibo.it
%%%%%%%%%%%%%%%%%%%%%%%%%%%%%%%%%%%%%%%%%%%%%%%%%%%%%%%%%%%%%%%%%%%%%%%%%%%%%%%%
%\documentclass[handout]{beamer}\mode<handout>{\usetheme{default}}
%
\documentclass[presentation]{beamer}\mode<presentation>{\usetheme{AMSBolognaFC}}
%\documentclass[handout]{beamer}\mode<handout>{\usetheme{AMSBolognaFC}}
%%%%%%%%%%%%%%%%%%%%%%%%%%%%%%%%%%%%%%%%%%%%%%%%%%%%%%%%%%%%%%%%%%%%%%%%%%%%%%%%
\usepackage{ise-lab-common}
\usepackage{ise-lab-perception-actuation}
% version
\newcommand{\versionmajor}{0}
\newcommand{\versionminor}{1}
\newcommand{\versionpatch}{1}
\newcommand{\version}{\versionmajor.\versionminor.\versionpatch}
%%%%%%%%%%%%%%%%%%%%%%%%%%%%%%%%%%%%%%%%%%%%%%%%%%%%%%%%%%%%%%%%%%%%%%%%%%%%%%%%
\title[\currentLab{} -- Perception \& Actuation]{Perception \& Actuation from the Agent Perspective}
%
\subtitle{\courseName{} / Module \moduleN{} (\courseAcronym)}
%
\author[\sspeaker{\gcShort}]{\speaker{\gcFull} \\ \gcEmail}
%
\institute[\disiShort, \uniboShort]{\disi{} (\disiShort)\\\unibo}
%
\date[A.Y. \academicYear{} (v.\ \version)]{Academic Year \academicYear{}\\(version \version)}
%
%%%%%%%%%%%%%%%%%%%%%%%%%%%%%%%%%%%%%%%%%%%%%%%%%%%%%%%%%%%%%%%%%%%%%%%%%%%%%%%%
\begin{document}
%%%%%%%%%%%%%%%%%%%%%%%%%%%%%%%%%%%%%%%%%%%%%%%%%%%%%%%%%%%%%%%%%%%%%%%%%%%%%%%%

%/////////
\frame{\titlepage}
%/////////

%%===============================================================================
\section*{Outline}
%%===============================================================================
%
%%/////////
\frame[c]{\tableofcontents[hideallsubsections]}
%%/////////

%===============================================================================
\section{Premises}
%===============================================================================

\begin{frame}{Lecture Goals}
    \begin{itemize}
        \item Understand how to enable the perception and actuation from the agent perspective
        \item Understand the notion of layered software system
        \item Understand the key role of Application Programming Interfaces (API)
    \end{itemize}
\end{frame}

%===============================================================================
\section{(Software) Environments}
%===============================================================================

\begin{frame}{Basics}
    \begin{itemize}
        \item 
    \end{itemize}
\end{frame}

%===============================================================================
\section{Demos}
%===============================================================================

\startDemo

\begin{frame}{\currentDemo{} -- First Demo}
    \begin{block}{Goal}
        Goal here
    \end{block}
    %
    \begin{itemize}
        \item further info here
    \end{itemize}
\end{frame}

%===============================================================================
\section{Exercises}
%===============================================================================

\startExercise

\begin{frame}{\currentExercise{} -- First Exercise}
    \begin{block}{Goal}
        Goal here
    \end{block}
    %
    \begin{itemize}
        \item further info here
    \end{itemize}
\end{frame}

%===============================================================================
\section*{}
%===============================================================================

%/////////
\frame{\titlepage}
%/////////

%===============================================================================
\section*{\refname}
%===============================================================================

%%%%
\setbeamertemplate{page number in head/foot}{}
%/////////
\begin{frame}[c,noframenumbering]{\refname}
%\begin{frame}[t,allowframebreaks,noframenumbering]{\refname}
%	\tiny
    \scriptsize
%	\footnotesize
    \bibliographystyle{apalike-AMS}
    \bibliography{ise-lab-perception-actuation}
\end{frame}
%/////////

%%%%%%%%%%%%%%%%%%%%%%%%%%%%%%%%%%%%%%%%%%%%%%%%%%%%%%%%%%%%%%%%%%%%%%%%%%%%%%%%
\end{document}
%%%%%%%%%%%%%%%%%%%%%%%%%%%%%%%%%%%%%%%%%%%%%%%%%%%%%%%%%%%%%%%%%%%%%%%%%%%%%%%%
