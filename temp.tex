\hypertarget{perception-actuation-from-the-agent-perspective}{%
\section{Perception \& Actuation from the Agent
Perspective}\label{perception-actuation-from-the-agent-perspective}}

\hypertarget{lecture-goals}{%
\subsection{Lecture Goals}\label{lecture-goals}}

\begin{itemize}
\item
  Understand how to enable the perception and actuation from the agent
  perspective
\item
  Understand the notion of layered software system
\item
  Understand the key role of Application Programming Interfaces
\end{itemize}

\hypertarget{software-environments}{%
\subsection{(Software) Environments}\label{software-environments}}

\begin{itemize}
\item
  Environment = anything laying outside the agents

  \begin{itemize}
  \tightlist
  \item
    Can be perceived via sensors (input)
  \item
    Can be affected via actuators (output)
  \end{itemize}
\item
  Software agents can be abstractly modelled as composed by layers

  \begin{enumerate}
  \def\labelenumi{\arabic{enumi}.}
  \tightlist
  \item
    Some control software (e.g.~Java program / logic program /
    AgentSpeak program)
  \item
    Some interpreter for that software (e.g.~JVM, Prolog interpreter,
    Jason)
  \item
    Some OS mediating the usage of HW resources for the interpreter
  \item
    Hardware resources

    \begin{itemize}
    \tightlist
    \item
      memory, storage, computational power, I/O (there including
      networking)
    \end{itemize}
  \end{enumerate}
\item
  From the sw agent's perspective, the environment is shaped by what
  perceptual / actuatory actions in can perform
\item
  The agent may also be endowed with \textbf{epistemic} capabilities

  \begin{itemize}
  \tightlist
  \item
    eg memorise new knowledge
  \item
    eg forget memorised knowledge
  \item
    eg update memorised knowledge
  \end{itemize}
\item
  Of course, endowing a sw agent with some perceptual / actuatory action
  requires

  \begin{itemize}
  \tightlist
  \item
    some enabling HW facility to be present
  \item
    some OS functionality enabling the usage of that HW facility
  \item
    the interpreter to know how to call that OS functionality
  \item
    the programming language to have some ad-hoc API wrapping the OS
    functionality
  \end{itemize}
\end{itemize}

\hypertarget{running-example}{%
\subsubsection{Running example}\label{running-example}}

The file system as an environment

\begin{itemize}
\item
  reading a file as a string, given its path as the perception action
\item
  writing a file as a string, given its path as the actuation action
\item
  capability to memorize custom information
\item
  capability to discard memorised information
\end{itemize}

\hypertarget{prolog-agents}{%
\subsection{Prolog Agents}\label{prolog-agents}}

\begin{itemize}
\item
  Prolog is goal-oriented
\item
  Basic structure of a Prolog agent:

\begin{Shaded}
\begin{Highlighting}[]
\NormalTok{start(}\DataTypeTok{Step}\NormalTok{) }\KeywordTok{:{-}} 
\NormalTok{    natural(}\DataTypeTok{Step}\NormalTok{)}\KeywordTok{,}
\NormalTok{    sleep(}\DecValTok{100} \CommentTok{/* ms */}\NormalTok{ )}\KeywordTok{,} \CommentTok{\% just to slow down the execution}
\NormalTok{    act(}\DataTypeTok{Step}\NormalTok{)}\KeywordTok{.}

\NormalTok{act(}\DataTypeTok{Step}\NormalTok{) }\KeywordTok{:{-}} \FunctionTok{write}\NormalTok{(}\StringTok{\textquotesingle{}}\ErrorTok{Hello world }\StringTok{\textquotesingle{}}\NormalTok{)}\KeywordTok{,} \FunctionTok{write}\NormalTok{(}\DataTypeTok{Step}\NormalTok{)}\KeywordTok{,} \FunctionTok{nl}\KeywordTok{.}
\end{Highlighting}
\end{Shaded}
\item
  Hence, it is very easy to see a Prolog solver as an agent capable of
  reasoning

  \begin{itemize}
  \tightlist
  \item
    Knowledge base -\textgreater{} memory
  \item
    Resolution -\textgreater{} thinking / reasoning
  \item
    Assertion/retraction -\textgreater{} epistemic actions
  \item
    Write, ??? -\textgreater{} Actuation
  \item
    ??? -\textgreater{} Perception
  \item
    ??? -\textgreater{} Environment
  \end{itemize}
\end{itemize}

\#\#\#~Example of thermostat agent in Prolog

\begin{Shaded}
\begin{Highlighting}[]

\NormalTok{warm\_range(}\DecValTok{20}\KeywordTok{,} \DecValTok{30}\NormalTok{)}\KeywordTok{.}

\NormalTok{start }\KeywordTok{:{-}} 
    \KeywordTok{repeat,}
\NormalTok{    sleep(}\DecValTok{100} \CommentTok{/* ms */}\NormalTok{)}\KeywordTok{,} \CommentTok{\% just to slow down the execution}
\NormalTok{    warm\_range(}\DataTypeTok{Min}\KeywordTok{,} \DataTypeTok{Max}\NormalTok{)}\KeywordTok{,}
\NormalTok{    keep\_temperature(}\DataTypeTok{Min}\KeywordTok{,} \DataTypeTok{Max}\NormalTok{)}\KeywordTok{.}

\NormalTok{keep\_temperature(}\DataTypeTok{Min}\KeywordTok{,} \DataTypeTok{Max}\NormalTok{) }\KeywordTok{:{-}} 
\NormalTok{    check\_temperature(}\DataTypeTok{T}\NormalTok{)}\KeywordTok{,}
\NormalTok{    handle\_temperature(}\DataTypeTok{T}\KeywordTok{,} \DataTypeTok{Min}\KeywordTok{,} \DataTypeTok{Max}\NormalTok{)}\KeywordTok{.}

\NormalTok{check\_temperature(}\DataTypeTok{T}\NormalTok{) }\KeywordTok{:{-}}
\NormalTok{    read\_text(}\OtherTok{"/}\ErrorTok{path}\OtherTok{/}\ErrorTok{to}\OtherTok{/}\ErrorTok{environment}\OtherTok{.}\ErrorTok{txt}\OtherTok{"}\KeywordTok{,} \DataTypeTok{T}\NormalTok{)}\KeywordTok{.}

\NormalTok{handle\_temperature(}\DataTypeTok{T}\KeywordTok{,} \DataTypeTok{Min}\KeywordTok{,} \DataTypeTok{\_}\NormalTok{) }\KeywordTok{:{-}} \DataTypeTok{T}\NormalTok{ \textless{}= }\DataTypeTok{Min}\KeywordTok{,} \KeywordTok{!,}
    \DataTypeTok{T1} \KeywordTok{=} \DataTypeTok{T} \FunctionTok{+} \DecValTok{1}\KeywordTok{,}
\NormalTok{    write\_text(}\OtherTok{"/}\ErrorTok{path}\OtherTok{/}\ErrorTok{to}\OtherTok{/}\ErrorTok{environment}\OtherTok{.}\ErrorTok{txt}\OtherTok{"}\KeywordTok{,} \DataTypeTok{T1}\NormalTok{)}\KeywordTok{.}

\NormalTok{handle\_temperature(}\DataTypeTok{T}\KeywordTok{,} \DataTypeTok{\_}\KeywordTok{,} \DataTypeTok{Max}\NormalTok{) }\KeywordTok{:{-}} \DataTypeTok{T} \DataTypeTok{\textgreater{}=} \DataTypeTok{Max}\KeywordTok{,} \KeywordTok{!,}
    \DataTypeTok{T1} \KeywordTok{=} \DataTypeTok{T} \FunctionTok{{-}} \DecValTok{1}\KeywordTok{,}
\NormalTok{    write\_text(}\OtherTok{"/}\ErrorTok{path}\OtherTok{/}\ErrorTok{to}\OtherTok{/}\ErrorTok{environment}\OtherTok{.}\ErrorTok{txt}\OtherTok{"}\KeywordTok{,} \DataTypeTok{T1}\NormalTok{)}\KeywordTok{.}

\NormalTok{handle\_temperature(}\DataTypeTok{\_}\KeywordTok{,} \DataTypeTok{\_}\KeywordTok{,} \DataTypeTok{\_}\NormalTok{)}\KeywordTok{.} \CommentTok{\% otherwise do nothing }
\end{Highlighting}
\end{Shaded}

\hypertarget{things-to-be-noticed}{%
\paragraph{Things to be noticed}\label{things-to-be-noticed}}

\begin{itemize}
\item
  Knowledge base -\textgreater{} memory
\item
  Resolution -\textgreater{} thinking / reasoning
\item
  Assertion/retraction -\textgreater{} epistemic actions
\item
  File writing -\textgreater{} Actuation
\item
  File reading -\textgreater{} Perception
\item
  File system -\textgreater{} Environment
\end{itemize}

\hypertarget{todo-list}{%
\paragraph{TODO list}\label{todo-list}}

\begin{itemize}
\tightlist
\item
  reading a file as a string, given its path as the perception action

  \begin{itemize}
  \tightlist
  \item
    read\_text/2 \textbf{(to be implemented)}
  \end{itemize}
\item
  writing a file as a string, given its path as the actuation action

  \begin{itemize}
  \tightlist
  \item
    write\_text/2 \textbf{(to be implemented)}
  \end{itemize}
\item
  capability to memorize custom information

  \begin{itemize}
  \tightlist
  \item
    assert/1, asserta/1, assertz/1 (provided)
  \end{itemize}
\item
  capability to discard memorised information

  \begin{itemize}
  \tightlist
  \item
    retract/1 (provided)
  \end{itemize}
\end{itemize}

\begin{quote}
How to realise these kind of functionalities?
\end{quote}

\begin{quote}
How to extend logic solver with further functionalities?
\end{quote}

\hypertarget{the-notion-of-generator}{%
\subsection{The notion of Generator}\label{the-notion-of-generator}}

\begin{quote}
citation to jelia + figure generator.svg
\end{quote}

Logic solver's gateways towards the world

Servers serving a solver's request to perform some functionality (and
providing 0, 1, or more responses)

\begin{itemize}
\tightlist
\item
  a particular case of artefact, under a MAS perspective
\end{itemize}

Definition:

\begin{itemize}
\item
  Signature: name + arity
\item
  Behaviour: function mapping \textbf{requests} to \textbf{\emph{a
  stream of} responses}

  \begin{itemize}
  \tightlist
  \item
    Positive responses
  \item
    Negative responses
  \item
    Exceptional responses
  \end{itemize}
\item
  Requests carry information about

  \begin{itemize}
  \tightlist
  \item
    the execution context (at the time of request issuing)
  \item
    the \emph{actual} arguments of the request
  \end{itemize}
\item
  Responses carry information about

  \begin{itemize}
  \tightlist
  \item
    success / failure / error
  \item
    substitution to be applied to variables
  \item
    side-effects to be applied to the execution context
  \end{itemize}
\end{itemize}

Workflow:

\begin{enumerate}
\def\labelenumi{\arabic{enumi}.}
\item
  When attempting to solve a goal \texttt{G} the solver may try to use a
  generator
\item
  Then it creates a request to be sent to the generator
\item
  The generator is then triggered: it produces a stream of responses
\item
  The solver lazily consumes the stream of responses

  \begin{itemize}
  \tightlist
  \item
    when a response is consumed depends on how the solver performs
    resolution
  \item
    whenever a response is consumed, any side effect possibly carried by
    that response is reified
  \end{itemize}
\item
  Each response constitutes a branch in the proof three, to be explored
  by the solver
\end{enumerate}

In pratice: 1. 2P-Kt provides the \texttt{Primitive} interface, to
implemented by generators - many sub-types are available in practice to
simplify programming generators

\begin{enumerate}
\def\labelenumi{\arabic{enumi}.}
\setcounter{enumi}{1}
\tightlist
\item
  Upon creation, solver accept \texttt{Libraries}, containing zero or
  more primitives

  \begin{itemize}
  \tightlist
  \item
    standard built-ins are constructed in this way
  \end{itemize}
\item
  During resolution, solvers may exploit generators to solve (sub-)goals
\end{enumerate}

\#\#\#~Examples

\begin{enumerate}
\def\labelenumi{\arabic{enumi}.}
\item
  \texttt{natural(?N)} tests or generates natural numbers
\item
  \texttt{update(?Fact)} retract \& update a fact in a single
  computational step
\end{enumerate}

\hypertarget{exercises}{%
\subsection{Exercises}\label{exercises}}

\begin{enumerate}
\def\labelenumi{\arabic{enumi}.}
\item
  \texttt{read\_text(+Path,\ -Text)} reads whole file as \texttt{Text},
  given its \texttt{Path}
\item
  \texttt{write\_text(+Path,\ +Text)} write \texttt{Text} as a whole
  file, given its \texttt{Path}

  \begin{itemize}
  \tightlist
  \item
    replaces the file silently if already present
  \end{itemize}
\item
  support smooth termination for the thermostat agent
\end{enumerate}

\#\#~Discussion about the thermostat agent

\begin{enumerate}
\def\labelenumi{\arabic{enumi}.}
\item
  Pro-active or reactive?
\item
  Autonomous?

  \begin{itemize}
  \tightlist
  \item
    w.r.t. motivational autonomy
  \item
    w.r.t. executive autonomy
  \item
    w.r.t. computational autonomy
  \end{itemize}
\item
  Weak goal or strong goal?
\end{enumerate}
